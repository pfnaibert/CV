%%%%%%%%%%%%%%%%%%%%%%%%%%%%%%%%%%%%%%%%%%%%%%%%%%%%%%%%%%%%%%
\documentclass[10pt,a4paper]{article}
\usepackage[a4paper, margin=2cm]{geometry}   % MARGENS

% =============================================================
% Basic Packages
\usepackage[T1]{fontenc}		% Selecao de codigos de fonte.
\usepackage[utf8]{inputenc}		% Codificacao do documento (conversão automática dos acentos)
\usepackage[brazilian]{babel}
\usepackage{microtype}
\usepackage{indentfirst}
\usepackage[colorlinks, citecolor=blue, urlcolor=blue, linkcolor=magenta]{hyperref} %
\usepackage{xcolor}
\usepackage{enumitem}
% \usepackage{fancyhdr}
% \pagestyle{fancy}

% =============================================================
% math packages
\usepackage{mathtools, amsmath, amssymb, amsthm, latexsym}

% =============================================================
% titling packages
\usepackage{titlesec}
\usepackage{titling}

% ==========================================================
% METADATA
\author{Paulo Ferreira Naibert}
\newcommand{\autor}{{Paulo Ferreira Naibert}}
\newcommand{\ufrgs}{{Universidade Federal do Rio Grande do Sul}}
%\newcommand{\ufrgs}{{Universidade Federal do Rio Gra}}

% ============================================================
% CONFIG

%--------------------------------------
\titleformat{\section}
{\bfseries\Large}
{}
{0em}
{}[\titlerule]

%--------------------------------------
\titleformat{\subsection}
{\bfseries\large}
{}
{0em}
{}[\vspace{-6pt}]

%--------------------------------------
\titleformat{\subsubsection}[runin]
{\bfseries}
{}
{0em}
{}[---]

% ============================================================
% maketitle
\renewcommand{\maketitle}{
\begin{center}
{\huge\bfseries\theauthor}
\\
% \today
\\
\vspace{.25em}
\textbf{e-mail:} \href{mailto:paulo.naibert@gmail.com}{paulo.naibert@gmail.com} |
\textbf{Linkedin:} \href{https://www.linkedin.com/in/paulo-naibert}{/paulo-naibert}
\\
\textbf{Github:} \href{https://github.com/pfnaibert}{/pfnaibert} |
\textbf{Lattes:} \href{http://lattes.cnpq.br/1511789633871437}{link}
\end{center}
}

% ============================================================
% Início do documento
\begin{document}
\pagenumbering{gobble}

% ============================================================
% ELEMENTOS PRÉ-TEXTUAIS
\maketitle

% ============================================================
% ELEMENTOS TEXTUAIS
% ============================================================
\vspace{-1.5 em}
\section{Resumo}
\noindent
Doutor em Economia Aplicada na UFRGS.
Foco de pesquisa em Mercado Financeiro e Carteiras de Investimento.
% Experiência em automatizar software (R) para a formação e avaliação de carteiras, incluindo visualização de medidas relevantes para carteiras de investimentos.
Experiência em programar e automatizar software (R) e análise de dados.


% ============================================================
\vspace{-1.5 em}
\section{Educação}

%--------------------------------------
\subsection{Doutorado em Economia Aplicada: 03/2015 -- 09/2019}
\ufrgs{}

Ênfase em análise de dados com R e MATLAB, estatísitca, econometria e finanças empíricas.

\textbf{Tese:} Essays in Portfolio Opmization

\textbf{Orientador:} João Frois Caldeira

%--------------------------------------
\vspace{-1 em}
\subsection{Mestrado em Economia Aplicada: 03/2013 -- 12/2014}
\ufrgs{}

\textbf{Dissertação:}
Seleção de carteiras com restrição da norma do vetor de alocação: uma aplicação a dados brasileiros.
\href{http://www.lume.ufrgs.br/handle/10183/132904}{Link para Dissertação}

\textbf{Orientador:} João Frois Caldeira

%--------------------------------------
\vspace{-1 em}
\subsection{Graduação em Ciências Econômicas: 03/2008 -- 12/2012}
\ufrgs{}

% Emphasis on statistics, econometrics, macroeconomics, microeconomics, mathematical economics, capital markets.

\textbf{TCC:}
Relação entre as volatilidades da inflação e do câmbio no Brasil (1999-2010)
\href{http://www.lume.ufrgs.br/handle/10183/38307}{Link para o TCC}

\textbf{Orientador:} Marcelo Savino Portugal

% ============================================================
\vspace{-1.5 em}
\section{Experiência}

%--------------------------------------
\subsection{Bolsista de Iniciação Científica 01/01/2010 -- 31/12/2012}
\textbf{Orientador:} Marcelo Savino Portugal

Responsabilidades: Auxílio na redação de relatórios sobre conjuntura macroeconômica.
Responsabilidades principais incluíam busca de informações relevantes sobre a economia nas bases de dados SGS-BACEN, SIDRA-IBGE, IPEA, etc.
Confecção de gráficos e tabelas para apresentação do relatório sobre conjuntura macroeconômica.

%--------------------------------------
\vspace{-1 em}
\subsection{Estágio de Docência}

\textbf{Orientador:} João Frois Caldeira \vspace{-6pt}
\begin{itemize}[noitemsep]

\item
Contabilidade Social (Graduação)

\item
Moderna Teoria do Portfólio (Mestrado)

\item
R aplicado a Moderna Teoria do Portfólio (Mestrado)
\end{itemize}

%--------------------------------------
\vspace{-1 em}
\subsection{Parecerista para Revistas Acadêmicas}

RBFin - Brazilian Review of Finance - 2016
\href{http://bibliotecadigital.fgv.br/ojs/index.php/rbfin/article/view/70380/67907}{Link}

% ============================================================
\vspace{-1.5 em}
\section{Publicações e Apresentações}

\begin{itemize}[noitemsep]
\item
Seleção de carteiras ótimas sob restrições nas normas dos vetores de alocação: uma avaliação empírica com dados da BM\&FBovespa
\hspace{2pt}
\href{http://bibliotecadigital.fgv.br/ojs/index.php/rbfin/article/view/52081}{Link to RBFin}
\hspace{2pt}
\href{https://www.anpec.org.br/encontro/2014/submissao/files_I/i8-4ddb23813d689d5ae684d5162e07302a.pdf}{Link to ANPEC 42}
\hspace{2pt}
\href{http://bibliotecadigital.fgv.br/ocs/index.php/ebf/15EBFin/paper/view/4971}{Link to EBFIN XV}

\item
Tangency Portfolio: A Critique
\hspace{2pt}
\href{http://sbfin.org.br/files/relatorio_de_artigos_final.pdf}{Link to EBFIN XIX}
\end{itemize}

% ============================================================
\vspace{-1.5 em}
\section{Informática}

\textbf{Markup:} \LaTeX, markdown, org, pandoc.

\textbf{Estatística e Econometria:} Eviews, R, MATLAB.

\textbf{Pacote Office:} Word, Excel, PowerPoint.

\textbf{Geral:} Linux, Bash.

\textbf{Editores de Texto:} Emacs, VIM.

% ============================================================
\vspace{-1.5 em}
\section{Idiomas}

\textbf{Portugês:} Língua Nativa.

\textbf{Inglês:} Fluente.

\textbf{Espanhol:} Leitura Intermediária.

% ============================================================
\end{document}
