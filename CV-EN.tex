%%%%%%%%%%%%%%%%%%%%%%%%%%%%%%%%%%%%%%%%%%%%%%%%%%%%%%%%%%%%%%
\documentclass[10pt, a4paper]{article}
\usepackage[a4paper, margin=2cm]{geometry} % margins

% =============================================================
% Basic Packages
\usepackage[T1]{fontenc}		% Selecao de codigos de fonte.
\usepackage[utf8]{inputenc}		% Codificacao do documento (conversão automática dos acentos)
\usepackage[brazilian]{babel}
\usepackage{microtype}
\usepackage{indentfirst}
\usepackage[colorlinks, citecolor=blue, urlcolor=blue, linkcolor=magenta]{hyperref} % 
\usepackage{xcolor}
\usepackage{enumitem}
% \usepackage{fancyhdr}
% \pagestyle{fancy}

% =============================================================
% math packages
\usepackage{mathtools, amsmath, amssymb, amsthm, latexsym}

% =============================================================
% titling packages
\usepackage{titlesec}
\usepackage{titling}

% ==========================================================
% METADATA
\author{Paulo Ferreira Naibert}
\newcommand{\autor}{{Paulo Ferreira Naibert}}
%\newcommand{\ufrgs}{{Universidade Federal do Rio Grande do Sul}}
\newcommand{\ufrgs}{{Federal University of Rio Grande do Sul}}

% ============================================================
% CONFIG

%--------------------------------------
\titleformat{\section}
{\bfseries\Large}
{}
{0em}
{}[\titlerule]

%--------------------------------------
\titleformat{\subsection}
{\bfseries\large}
{}
{0em}
{}[\vspace{-1 ex}]

%--------------------------------------
\titleformat{\subsubsection}[runin]
{\bfseries}
{}
{0em}
{}[---]

% ============================================================
% maketitle
\renewcommand{\maketitle}{
\begin{center}
{\huge\bfseries\theauthor}
\\
% \today
\\
\vspace{.25em}
\textbf{e-mail:} \href{mailto:paulo.naibert@gmail.com}{paulo.naibert@gmail.com} |
\textbf{Linkedin:} \href{https://www.linkedin.com/in/paulo-naibert}{/paulo-naibert}
\\
\textbf{Github:} \href{https://github.com/pfnaibert}{/pfnaibert} |
\textbf{Lattes:} \href{http://lattes.cnpq.br/1511789633871437}{link}
\end{center}
}

% ============================================================
% Início do documento
\begin{document}
\pagenumbering{gobble}

% ============================================================
% ELEMENTOS PRÉ-TEXTUAIS
\maketitle

% ============================================================
% ELEMENTOS TEXTUAIS
% ============================================================
\vspace{-12pt}
\section{Description}
PhD in Applied Economics at UFRGS.
My research focused mainly in Stock and Bond Markets.
In it, I used quantitative methods to form optimal portfolio of those different asset classes.
I have experience in automating software (R) to form and assess portfolios, inculding visualization of relevant Portfolio measures.

% ============================================================
\vspace{-12pt}
\section{Education}

%--------------------------------------
\subsection{Doctorate: 03/2015 -- 09/2020}
\textbf{Academic Doctorate in Applied Economics}
\ufrgs{}

Emphasis on data analysis with R and MATLAB, statistics, econometrics and empirical finance.

\textbf{Thesis:} Essays in Portfolio Optimization

%--------------------------------------
\vspace{-12pt}
\subsection{Master's Degree: 03/2013 -- 03/2015}
\textbf{Academic Master in Applied Economics}
\ufrgs{}

% Emphasis on data analysis with R and MATLAB, statistics, econometrics, empirical finance, economic modelling, computational methods applied to economics, macroeconomics, microeconomics.

\textbf{Dissertation (In Portuguese):}
Portfolio Selection under norm constraints: An Analysis of Brazilian Data.
\href{http://www.lume.ufrgs.br/handle/10183/132904}{Link to Dissertation}

%--------------------------------------
\vspace{-12pt}
\subsection{Bachelor: 03/2008 -- 12/2012}
\textbf{Bachelor in Economic Science}
\ufrgs{}

% Emphasis on statistics, econometrics, macroeconomics, microeconomics, mathematical economics, capital markets.

\textbf{Graduate Thesis (In Portuguese):}
Volatility Relationship Between Inflation and Currency Exchange in Brazil (1999-2010)
\href{http://www.lume.ufrgs.br/handle/10183/38307}{Link to Thesis}

% ============================================================
\vspace{-6pt}
\section{Experience}

%--------------------------------------
\subsection{Research Assistant: 01/01/2010 -- 31/12/2012}
Auxiliary role composing reports about macroeconomic situation.
One of the key responsabilities was the search of relevant information about the Economy (mainly Brazilian) on databases such as SGS-BACEN, SIDRA-IBGE, IPEA, etc.
Making Graphs and Tables for the report presentation about the macroeconomic situation.

%--------------------------------------
\vspace{-12pt}

\subsection{Teaching Assitant}
\begin{itemize}[noitemsep]

\item
National Accounts (Undergraduate Level)

\item
Modern Portfolio Theory (for Master's Students)

\item
R applied to Modern Portfolio Theory (for Master's Students)
\end{itemize}

%--------------------------------------
\vspace{-12pt}
\subsection{Referee for Academic Journals}
RBFin -- Brazilian Review of Finance -- 2016
\href{http://bibliotecadigital.fgv.br/ojs/index.php/rbfin/article/view/70380/67907}{Link}

% ============================================================
\vspace{-6pt}
\section{Publications and Presentations}

\begin{itemize}[noitemsep]

\item
Selection of optimal portfolios under norm constraints in the allocation vectors: an empirical evaluation with data from BM\&FBovespa (In Portuguese):
\hspace{2pt}
\href{http://bibliotecadigital.fgv.br/ojs/index.php/rbfin/article/view/52081}{Link to RBFin}
\hspace{2pt}
\href{https://www.anpec.org.br/encontro/2014/submissao/files_I/i8-4ddb23813d689d5ae684d5162e07302a.pdf}{Link to ANPEC 42}
\hspace{2pt}
\href{http://bibliotecadigital.fgv.br/ocs/index.php/ebf/15EBFin/paper/view/4971}{Link to EBFIN XV}

\item
Tangency Portfolio: A Critique
\href{http://sbfin.org.br/files/relatorio_de_artigos_final.pdf}{Link to EBFIN XIX}
\end{itemize}

% ============================================================
\vspace{-12pt}
\section{Computer Skills}

\textbf{Markup} \LaTeX, markdown, org, pandoc.

\textbf{Statistics and Econometrics}: Eviews, R, MATLAB.

\textbf{Office Suite}: Word, Excel, PowerPoint.

\textbf{General}: Linux, Bash.

\textbf{Text Editors}: Emacs, VIM.

% ============================================================
\vspace{-6pt}
\section{Languages}

\textbf{Portuguese}: Native Language

\textbf{English}: Fluent in Reading, Talking and Listening. Advanced Writing.

\textbf{Spanish}: Advanced Reading.

% ============================================================
\end{document}

